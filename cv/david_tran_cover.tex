%%%%%%%%%%%%%%%%%%%%%%%%%%%%%%%%%%%%%%%%%
% "ModernCV" CV and Cover Letter
% LaTeX Template
% Version 1.1 (9/12/12)
%
% This template has been downloaded from:
% http://www.LaTeXTemplates.com
%
% Original author:
% Xavier Danaux (xdanaux@gmail.com)
%
% License:
% CC BY-NC-SA 3.0 (http://creativecommons.org/licenses/by-nc-sa/3.0/)
%
% Important note:
% This template requires the moderncv.cls and .sty files to be in the same
% directory as this .tex file. These files provide the resume style and themes
% used for structuring the document.
%
%%%%%%%%%%%%%%%%%%%%%%%%%%%%%%%%%%%%%%%%%

%----------------------------------------------------------------------------------------
%	PACKAGES AND OTHER DOCUMENT CONFIGURATIONS
%----------------------------------------------------------------------------------------

\documentclass[11pt,a4paper,sans]{moderncv} % Font sizes: 10, 11, or 12; paper sizes: a4paper, letterpaper, a5paper, legalpaper, executivepaper or landscape; font families: sans or roman

\moderncvstyle{casual} % CV theme - options include: 'casual' (default), 'classic', 'oldstyle' and 'banking'
\moderncvcolor{black} % CV color - options include: 'blue' (default), 'orange', 'green', 'red', 'purple', 'grey' and 'black'

\usepackage{lipsum} % Used for inserting dummy 'Lorem ipsum' text into the template

\usepackage[scale=0.75]{geometry} % Reduce document margins
%\setlength{\hintscolumnwidth}{3cm} % Uncomment to change the width of the dates column
%\setlength{\makecvtitlenamewidth}{10cm} % For the 'classic' style, uncomment to adjust the width of the space allocated to your name

%----------------------------------------------------------------------------------------
%	NAME AND CONTACT INFORMATION SECTION
%----------------------------------------------------------------------------------------

\firstname{David} % Your first name
\familyname{Tran} % Your last name

% All information in this block is optional, comment out any lines you don't need
\title{Research Statement}
% \address{W. Ethan Eagle}{}
% \mobile{(302) 584 3464}
% \phone{(000) 111 1112}
% \fax{(000) 111 1113}
\email{davidtranhq@gmail.com}
\homepage{davidtranhq.github.io}
% \homepage{staff.org.edu/~jsmith}{staff.org.edu/$\sim$jsmith} % The first argument is the url for the clickable link, the second argument is the url displayed in the template - this allows special characters to be displayed such as the tilde in this example
% \extrainfo{additional information}
% \photo[70pt][0.4pt]{pictures/picture} % The first bracket is the picture height, the second is the thickness of the frame around the picture (0pt for no frame)
% \quote{"A witty and playful quotation" - John Smith}


%----------------------------------------------------------------------------------------

\begin{document}
\makecvtitle % Print the CV title


\setlength{\parskip}{12pt}

%----------------------------------------------------------------------------------------
%	EDUCATION SECTION
%----------------------------------------------------------------------------------------

My broad research interest revolves around pushing the boundaries of what can be practically computed.

I am entering the final year of my Bachelor's with a dual in Honors Computer Science and a Major in Mathematics and currently hold the highest standing in the Department of Computer Science. I have experience in machine learning and computational neuroscience research: I was an undergraduate researcher at Western's Brain and Mind Institute under the supervision of Dr. Jörn Diedrichsen with funding from NSERC's USRA. We developed a novel architecture for generating functional parcellations of the cerebellum using fMRI data, which involved conducting research in problems related to Bayesian unsupervised learning, approximate infererence, and hidden Markov models. Currently, I am a part of my department's Quantum Computing Research Group, and am conducting research in quantum information theory, in particular in quantum error-correcting codes.

Apart from research, I also have professional experience in software engineering. Most recently, I was a software engineer in Silicon Valley at Snowflake, a software company specializing in high-performing and highly-scalable distributed computing. Even as an engineer, my interests lean toward research work: at Snowflake the body of my work was mainly exploratory and experimental, involving the discovery of optimizations for the perfomance and sources of pessimizations of the execution platform.

\end{document}