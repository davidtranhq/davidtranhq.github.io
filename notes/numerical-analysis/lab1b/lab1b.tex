\documentclass[letter,11pt]{article}

\usepackage{titlesec}
\usepackage{graphicx}
\usepackage{caption}
\usepackage{subcaption}
\usepackage{amsmath}
\usepackage{amsfonts}
\usepackage{amssymb}
\usepackage{hyperref}
\usepackage{enumitem}

% Adjust the margins if needed
\usepackage[left=1in, right=1in, top=1in, bottom=1in]{geometry}

% Set the title and author
\title{Iteration and Error}
\author{David Tran and Spencer Kelly}
\date{\today}

\begin{document}

\maketitle

\subsection*{Abstract}

\subsection*{Introduction}

\subsection*{Particular aspects of this Lab }

\subsection*{Taylor series and error}
\subsubsection*{Problem 0}
Let $f(x) = e^{-x}$. Note that $f^{(1)}(x) = -e^{-x}, f^{(2)}(x) = e^{-x}$, and in general, $f^{(n)}(x) = (-1)^ne^{-x}$. So, $f^{(n)}(0) = (-1)^n$, and the Taylor series expansion around $x_0 = 0$ is given by

\begin{align*}
T_n(x) &= \sum_{k = 0}^n \frac{f^{(k)}(x_0)}{k!} x^k \\
&= \sum_{k = 0}^n \frac{(-1)^k}{k!}x^k
\end{align*}

Similarily, the remainder term is

\begin{align*}
R_n &= \frac{f^{(n + 1)}(z)}{(n + 1)!}(x - x_0)^{n + 1} \\
&= \frac{(-1)^{n + 1} e^{-z}}{(n + 1)!}x^{n + 1}.
\end{align*}

for some $0 \leq z \leq x$.

\subsubsection*{Problem 1}
\begin{enumerate}[label=\alph*.]
  \item We have \begin{align*}
    e^{-x} &= 1 - x + \frac{x^2}{2!} - \frac{x^3}{3!} + \dots
  \end{align*}
  so
  \begin{align*}
    F(x) &= \frac{e^{-x} - 1 + x}{x^2} \\
    &= \frac{1}{2!} - \frac{x}{3!} + \frac{x^2}{4!} - \dots \\
    &= \sum_{k = 0}^n (-1)^k \frac{x^k}{(k + 2)!}.
  \end{align*}

  The remainder term is TODO.
  \item TODO.
\end{enumerate}

\subsubsection*{Problem 2}
\begin{enumerate}[label=\alph*.]
  \item TODO.
  \item TODO.
\end{enumerate}

\subsubsection*{Problem 3}
\begin{enumerate}[label=\alph*.]
  \item TODO.
  \item TODO.
  \item TODO.
  \item TODO.
\end{enumerate}

\subsubsection*{Problem 4}
\begin{enumerate}[label=\alph*.]
  \item TODO.
  \item TODO.
  \item TODO.
\end{enumerate}

\subsection*{Summary and Conclusions}

\begin{thebibliography}{99}
  % Add your references here using the \bibitem{} command
  % Example: \bibitem{author_year} Author, A., \& Year, Y. Title of the paper. \textit{Journal Name}, \textbf{Volume}(Issue), PageRange.
\end{thebibliography}

\subsection*{Teamwork Statement}

\subsection*{Code Appendix}

\subsection*{Plot Appendix}


\end{document}
