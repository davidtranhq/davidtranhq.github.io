\documentclass[letter,11pt]{article}

\usepackage{titlesec}
\usepackage{graphicx}
\usepackage{caption}
\usepackage{subcaption}
\usepackage{amsmath}
\usepackage{amsfonts}
\usepackage{amssymb}
\usepackage{hyperref}
\usepackage{enumitem}
\usepackage{listings}
\usepackage{xcolor}

% Define colors for syntax highlighting
\definecolor{mygreen}{rgb}{0,0.6,0}
\definecolor{mygray}{rgb}{0.5,0.5,0.5}
\definecolor{mymauve}{rgb}{0.58,0,0.82}

% Set up the MATLAB code listing style
\lstset{
  backgroundcolor=\color{white},
  basicstyle=\footnotesize\ttfamily,
  breakatwhitespace=false,
  breaklines=true,
  captionpos=b,
  commentstyle=\color{mygreen},
  deletekeywords={...},
  escapeinside={\%*}{*)},
  extendedchars=true,
  frame=single,
  keepspaces=true,
  keywordstyle=\color{blue},
  language=Matlab,
  otherkeywords={*,...},
  numbers=left,
  numbersep=5pt,
  numberstyle=\tiny\color{mygray},
  rulecolor=\color{black},
  showspaces=false,
  showstringspaces=false,
  showtabs=false,
  stepnumber=1,
  stringstyle=\color{mymauve},
  tabsize=2,
  title=\lstname
}


% Adjust the margins if needed
\usepackage[left=1in, right=1in, top=1in, bottom=1in]{geometry}

% Set the title and author
\title{Iteration and Error}
\author{David Tran and Spencer Kelly}
\date{\today}

\begin{document}

\maketitle

\subsection*{Abstract}
The following series of labs explore different algorithms in numerical analysis, as well as their applications. We offer an exploratory look on each of the algorithms, including notes on its implementation, properties of the algorithms, and the practical effects of its implementation on real-world systems.

This lab introduces the concept of iteration in numerical algorithms through use of Taylor series. Through computations, we show how well the Taylor series approximates a function, and additionally quantify the error in its approximation. We observe and discuss the effects of truncation and precision or rounding errors which occur due to the limitations of hardware, and show how we can measure and differentiate between the two.

\section{Introduction}

\section{Taylor series and error}

\setcounter{subsection}{-1}
\subsection{Problem 0}
Let $f(x) = e^{-x}$. Note that $f^{(1)}(x) = -e^{-x}, f^{(2)}(x) = e^{-x}$, and in general, $f^{(n)}(x) = (-1)^ne^{-x}$. So, $f^{(n)}(0) = (-1)^n$, and the Taylor series expansion around $x_0 = 0$ is given by

\begin{align*}
T_n(x) &= \sum_{k = 0}^n \frac{f^{(k)}(x_0)}{k!} x^k \\
&= \sum_{k = 0}^n \frac{(-1)^k}{k!}x^k
\end{align*}

THIS IS SPENCER;

Similarily, the remainder term is

\begin{align*}
R_n &= \frac{f^{(n + 1)}(z)}{(n + 1)!}(x - x_0)^{n + 1} \\
&= \frac{(-1)^{n + 1} e^{-c}}{(n + 1)!}x^{n + 1}.
\end{align*}

for some $0 \leq c \leq x$.

\subsection{Problem 1}
\begin{enumerate}[label=\alph*.]
  \item We have \begin{align*}
    e^{-x} &= 1 - x + \frac{x^2}{2!} - \frac{x^3}{3!} + \dots
  \end{align*}
  so
  \begin{align*}
    F(x) &= \frac{e^{-x} - 1 + x}{x^2} \\
    &= \frac{1}{2!} - \frac{x}{3!} + \frac{x^2}{4!} - \dots \\
    &= \sum_{k = 0}^\infty (-1)^k \frac{x^k}{(k + 2)!}
  \end{align*}

  and its Taylor series up to $n + 1$ terms is

  $$
  T_n(x) = \sum_{k = 0}^n (-1)^k \frac{x^k}{(k + 2)}
  $$

  To find the remainder term $R_n$, we use the fact that $R_n = F(x) - T_n(x)$. 
  \item We know that the Taylor remainder is in general written as
  $$\frac{f^{k+1}(c)}{(k+1)!}(x-x_0)^{k+1}$$

 So that, in this case, we end up with our taylor series of $n+1$ terms having remainder term:
 $$\frac{(-1)^{n+1}}{((n+1)+2)!}(x)^{n+1} = \frac{(-1)^{n+1}}{(n+3)!}(x)^{n+1}$$

 This tells us that the bound for the truncation error can be written as

 $$\frac{(|x|)^{n+1}}{((n+1)+2)!}$$
\end{enumerate}

\newpage
\subsection{Problem 2}
We compute and plot $F(x)$ from above with the following.
  
  \lstinputlisting{FTaylor.m}

  \lstinputlisting{plotF.m}

giving the graph in Figure~\ref{fig:lab1b_2}:

\begin{figure}[h]
  \centering
  \includegraphics[width=0.8\linewidth]{lab1b_2.png}
  \caption{$F(x)$ and its Taylor Series from $(-0.01, 0.01)$}
  \label{fig:lab1b_2}
\end{figure}

\subsection{Problem 3}

Plotting the graph from Problem 2 using instead a range of $(-10^{-7}, 10^{-7})$ yields Figure~\ref{fig:lab1b_3}. Observe the absolute and relative errors over the interval $(-10^{-6}, 10^{-6})$ in Figure~\ref{fig:lab1b_3_error}. We use the truncation error term in Problem 1 as an upper bound on truncation error. If we plot this alongside the absolute and relative errors as in Figure~\ref{fig:lab1b_3_error_bound}, we see the absolute and relative errors are well above the theoretical truncation error. Also note that the truncation error decreases as we approach 0 and increases as we move away from it, since our Taylor series expansion is centered at $x = 0$. Because the absolute and relative errors are above the truncation error, the problem is roundoff error rather than truncation error. This is because the order of magnitude of the values we are working with ($10^[-6]$) are so small that we approach the machine epsilon when we take powers of these values. The limited precision of floating-point numbers mean that we can't represent these values as precisely. Indeed, as we get small enough, we can \textit{underflow}: where the number is smaller than our floating-point representation can represent. This lack of floating-point precision is what is making the roundoff errors so large.

So, the directly computed function is more accurate. We know the closed form of $F(x)$, so we should use it, rather than approximate the function with the terms of a Taylor series, since, at values of such a small magnitude, it introduces more roundoff error.

\begin{figure}[h]
  \centering
  \includegraphics[width=0.8\linewidth]{lab1b_3.png}
  \caption{$F(x)$ and its Taylor Series from $(-10^{-7}, 10^{-7})$ up to cubic terms}
  \label{fig:lab1b_3}
\end{figure}

\begin{figure}[h]
  \centering
  \includegraphics[width=0.8\linewidth]{lab1b_3_error.png}
  \caption{Error between $F(x)$ and its Taylor Series}
  \label{fig:lab1b_3_error}
\end{figure}

\begin{figure}[h]
  \centering
  \includegraphics[width=0.8\linewidth]{lab1b_3_error_bound.png}
  \caption{Error between $F(x)$ and its Taylor Series and truncation bound}
  \label{fig:lab1b_3_error_bound}
\end{figure}

\section{Summary and Conclusions}

In this lab, we used the Taylor series the approximate a function whose closed-form we knew. We plotted the approximation alongside the closed form to observe how well the Taylor series was able to approximate, as well as where it fell short. We found both analytically and empirically the expected error due to truncation of the series, and discovered that roundoff errors dominated the error over truncation errors at very small values of $x$.

The method of using a Taylor series to approximate a function is very applicable and often used to compute values for functions whose closed form we do not know. For example, in many programming languages, the trigonometric functions are defined in terms of their power series, and 3 to 4 terms are usually used as enough to achieve the maximum-allowed precision due to floating-point representation. For future work, it wouuld be good to explore the threshold at which roundoff error begins to dominate truncation error, as well as discover possible workarounds for approximating functions with Taylor series at very small input values.

\begin{thebibliography}{99}
  % Add your references here using the \bibitem{} command
  % Example: \bibitem{author_year} Author, A., \& Year, Y. Title of the paper. \textit{Journal Name}, \textbf{Volume}(Issue), PageRange.
\end{thebibliography}

\section*{Teamwork Statement}

\section*{Code Appendix}

\section*{Plot Appendix}


\end{document}
