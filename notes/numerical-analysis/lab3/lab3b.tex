\documentclass[11pt]{article}

\usepackage{titlesec}
\usepackage{graphicx}
\usepackage{caption}
\usepackage{subcaption}
\usepackage{amsmath}
\usepackage{amsfonts}
\usepackage{amssymb}
\usepackage{hyperref}
\usepackage{enumitem}
\usepackage{listings}
\usepackage{xcolor}

% Define colors for syntax highlighting
\definecolor{mygreen}{rgb}{0,0.6,0}
\definecolor{mygray}{rgb}{0.5,0.5,0.5}
\definecolor{mymauve}{rgb}{0.58,0,0.82}

% Set up the MATLAB code listing style
\lstset{
  backgroundcolor=\color{white},
  basicstyle=\footnotesize\ttfamily,
  breakatwhitespace=false,
  breaklines=true,
  captionpos=b,
  commentstyle=\color{mygreen},
  deletekeywords={...},
  escapeinside={\%*}{*)},
  extendedchars=true,
  frame=single,
  keepspaces=true,
  keywordstyle=\color{blue},
  language=Matlab,
  otherkeywords={*,...},
  numbers=left,
  numbersep=5pt,
  numberstyle=\tiny\color{mygray},
  rulecolor=\color{black},
  showspaces=false,
  showstringspaces=false,
  showtabs=false,
  stepnumber=1,
  stringstyle=\color{mymauve},
  tabsize=2,
  title=\lstname
}


% Adjust the margins if needed
\usepackage[left=1in, right=1in, top=1in, bottom=1in]{geometry}
\usepackage{graphicx}
\usepackage{graphicx}
\usepackage{tabto}

% Set the title and author
\title{Solving Systems of Equations, Errors and Explorations}
\author{David Tran and Spencer Kelly}
\date{\today}

\begin{document}

\maketitle

\subsection*{Abstract}

\section{Introduction}

\section{The PA = LU factorization method for linear systems}

\subsection{Why is PA = LU needed for solving linear systems approximately?}

When solving linear systems of the form $Ax = b$, we begin by gaussian elimination of the matrix $A$, followed by back substitution, and ultimately arrive at our solution.
However, when a particular matrix $A$ is being used for multiple iterations, the overhead involved can become quite an obstacle.
This is because the process of Gaussian elimination is a computationally expensive process, with complexity on the order $O(n^2)$.
But with $PA = LU$ factorization, we essentially remove the overhead involved with gassian elimination, for all but the first iteration, by rewriting the matrix $A$ in terms of the upper and lower matrices $L$, and $U$, respectively.
Thus, for every subsequent iteration involving the same matrix, we need not perform gaussian elimination, since $L$ and $U$ are all we need to begin performing back-substitution, which has complexity $O(n)$.

\subsection{How to identify systems Ax = b for which PA = LU is not suited}

\subsection{Larger applications of PA = LU factorization}

\section{Iterative solution of systems of linear equations}

\subsection{Solving an equation for n = 100,000}

\subsection{Comparison of PA = LU and Jacobi Iteration}

\subsection{Why is solving such large systems important in applications?}

\section{Implement Newton's method for multiple variables}

\subsection{Implement Newton's method for systems using vectorization}

\subsection{Testing}

\subsection{Challenging Example}

\section{Summary}

\section{Appendices}

\subsection{Code}

\subsection{Plots}





\section{Code}



\section{Summary}
\subsection{Results}


\subsection{Team Description}


\subsection{Future Explorations}


\subsection{References}

\section*{Appendix}


\end{document}
