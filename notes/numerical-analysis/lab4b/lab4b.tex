\documentclass[11pt]{article}

\usepackage{titlesec}
\usepackage{graphicx}
\usepackage{caption}
\usepackage{subcaption}
\usepackage{amsmath}
\usepackage{amsfonts}
\usepackage{amssymb}
\usepackage{hyperref}
\usepackage{enumitem}
\usepackage{listings}
\usepackage{xcolor}

% Define colors for syntax highlighting
\definecolor{mygreen}{rgb}{0,0.6,0}
\definecolor{mygray}{rgb}{0.5,0.5,0.5}
\definecolor{mymauve}{rgb}{0.58,0,0.82}

% Set up the MATLAB code listing style
\lstset{
  backgroundcolor=\color{white},
  basicstyle=\footnotesize\ttfamily,
  breakatwhitespace=false,
  breaklines=true,
  captionpos=b,
  commentstyle=\color{mygreen},
  deletekeywords={...},
  escapeinside={\%*}{*)},
  extendedchars=true,
  frame=single,
  keepspaces=true,
  keywordstyle=\color{blue},
  language=Matlab,
  otherkeywords={*,...},
  numbers=left,
  numbersep=5pt,
  numberstyle=\tiny\color{mygray},
  rulecolor=\color{black},
  showspaces=false,
  showstringspaces=false,
  showtabs=false,
  stepnumber=1,
  stringstyle=\color{mymauve},
  tabsize=2,
  title=\lstname
}


% Adjust the margins if needed
\usepackage[left=1in, right=1in, top=1in, bottom=1in]{geometry}
\usepackage{graphicx}
\usepackage{graphicx}
\usepackage{tabto}

% Set the title and author
\title{Neural Ordinary Differential Equations}
\author{David Tran and Spencer Kelly}
\date{\today}

\begin{document}

\maketitle

\section*{Abstract}

\section{Introduction}

\section{Differential Equation Solvers}

In this section, we discuss two different methods of numerical differential equation solving, their accuracy, and their importance.

\subsection{Euler's Method}

\subsection{Runge-Kutta Methods}

\section{Neural Ordinary Differential Equations}

\subsection{What is a neural ODE?}

\subsection{Why machine learning for DE solving}

The advantage of machine learning for DE solving over traditional analytical or other numerical approximation methods is due to their flexibility in approximating relations of arbitrary complexity. Due to the performance of the model being a function of the amount of data available on the relation-of-interest, neural networks are particularly advantageous for solving differential equations for which the knowledge of the underlying dynamics of the relation are unknown or limited, compared to the large amount of data representing the relation.

\subsection{Application}

We use the implementation of the neural ODE described in \cite{chen2018neuralode} using the code in \cite{torchdiffeq}. We use it to learn the dynamics of a simple harmonic oscillator with slight dampening. Observe in Figure~\ref{fig:first_iteration} how the predicted trajectories and phase portrait (blue) do not match very well the ground truth (green). Although the shape of the learned vector field looks accurate, it is askew from the proper orientation of the phase portrait In Figure~\ref{fig:last_iteration}, the predicted trajectory and phase portrait nearly perfectly coincide, and we observe that the learned vector field nearly matches what one would expect from the phase portrait.

\begin{figure}
  \centering
  \includegraphics*[width=\linewidth]{000.png}
  \caption{The predicted trajectory, and the corresponding learned vector field after one iteration. The green represents the ground truth, while the blue represents the output of the model.}
  \label{fig:first_iteration}
\end{figure}

\begin{figure}
  \centering
  \includegraphics*[width=\linewidth]{099.png}
  \caption{After 99 iterations.}
  \label{fig:last_iteration}
\end{figure}

\subsubsection{Euler's Method vs Runge-Kutta}

\begin{thebibliography}{9}

  \bibitem{chen2018neuralode}
    Chen, R. T. Q., Rubanova, Y., Bettencourt, J., \& Duvenaud, D. (2018).
    \textit{Neural Ordinary Differential Equations}.
    Advances in Neural Information Processing Systems.
  
  \bibitem{torchdiffeq}
    Chen, R. T. Q. (2018).
    \textit{torchdiffeq}.
    Retrieved from \url{https://github.com/rtqichen/torchdiffeq}
  
\end{thebibliography}



  


\end{document}
