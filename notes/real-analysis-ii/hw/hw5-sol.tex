\documentclass[11pt, letterpaper]{article}
\usepackage{fullpage}
\usepackage{amsmath,amsthm,amsfonts,amssymb,amscd}
\usepackage{lastpage}
\usepackage{enumerate}
\usepackage{fancyhdr}
\usepackage{mathrsfs}
\usepackage{enumitem} % \setlist
% for \imageans: float for [H] so the figure floats
\usepackage{graphicx}
\usepackage{adjustbox}
\usepackage{float} 

\setlength{\parindent}{0.25in}
\setlength{\parskip}{0.05in}
% indent paragraphs in list
\setlist{  
  listparindent=\parindent,
  parsep=0pt,
}

% Include graphics in answer
\newcommand{\imageans}[1]
{%
    \begin{figure}[H]
        \centering
        \includegraphics[width=0.4\linewidth]{#1}
    \end{figure}
}

% comments inside align environnment
\newcommand{\comment}[1]{%
  \text{\phantom{(#1)}} \tag{#1}
}
\newtheorem{theorem}{Theorem}
\newtheorem{lemma}{Lemma}
% Cases for Proof environment
\newlist{pcases}{enumerate}{1}
\setlist[pcases]{
  label=\underline{Case~\arabic*}:\protect\thiscase.~,
  ref=\arabic*,
  align=left,
  labelsep=0pt,
  leftmargin=0pt,
  labelwidth=0pt,
  parsep=0pt
}
\newcommand{\case}[1][]{%
  \if\relax\detokenize{#1}\relax
    \def\thiscase{}%
  \else
    \def\thiscase{~#1}%
  \fi
  \item
}

% Edit these as appropriate
\newcommand\course{Math 3122}
\newcommand\hwtitle{HW 5}                  
\newcommand\name{David Tran}
\newcommand\studentid{251169871}

\fancypagestyle{firststyle}
{
    \headheight 35pt
    \lhead{\name}
    \lhead{\name\\\studentid}
    \chead{\textbf{\LARGE \hwtitle}}
    \rhead{\course \\ \today}
    \lfoot{}
    \cfoot{}
    \rfoot{\small\thepage}
    \headsep 1.5em
}

\DeclareUnicodeCharacter{2212}{-}
\begin{document}

\thispagestyle{firststyle}

\setlist[enumerate]{leftmargin=*} % remove enuemrate indentation

\begin{enumerate}
  \item \begin{proof}
    Let $U$ be an open cover of $X$. Since $U$ is open, for every $x \in U$, there is an open ball $B_x$ of radius $\epsilon(x)$ around $x$ contained in $U$. Consider the union of all such balls $V = \bigcup_{x \in U} B_x$. Since every ball is contained in $U$, $V \subseteq U$. Also, since every $x \in U$ is contained in some ball $B_x$, $U \subseteq V$. Thus, $U = V$, so $V$ is a cover of $X$ by open balls, so it has a finite subcover. So $U$ has a finite subcover, so $X$ is compact.
  \end{proof}

  \item \begin{proof}
    Suppose $X$ is not compact. Let $(x_n)$, $n \in \mathbb N$ be an arbitrary sequence with no accumulation point in $X$. Define $f(x) = \inf_n \lbrace d(x, x_n) + 1/n \rbrace$. First we show that $f$ is continuous. Let $x, y \in X$ with $d(x, y) > 0$. We have that for $n \in \mathbb N$,
    $$f(x) \leq d(x, x_n) + 1/n \leq d(x, y) + d(y, x_n) + 2/n$$
    so $f(x) \leq d(x, y) + f(y) + 1/n$. Similarily, $f(y) \leq d(x, y) + f(x) + 1/n$. So,
    $$
    \vert f(x) - f(y) \vert \leq d(x, y) + 1/n
    $$
    Since $1/n$ and $d(x, y)$ can be made arbitrarily small, $f$ is continuous.

    Next we show that $f(X)$ is non-compact. Note that $0 \not\in f(X)$ since if $x \not\in (x_n)$, then $f(x) > 0$ since $x$ is not an accumulation point, so $\lim_{n \to \infty} d(x, x_n) \neq 0$. If instead $x \in (x_n)$, then
    $$f(x) = \inf \lbrace 1/n : x_n = x \rbrace = \min \lbrace 1/n : x_n \rbrace > 0$$ 
    since there are finitely many $x_n = x$. We know there are only finitely many such $x_n$ since if there weren't, $(x_n)$ would have an accumulation point $x$. Finally, note that $1/n \in f(X)$ for arbitrarily large $n$ since $f(y) = 1/n$, where $n = \max\lbrace n : y = x_n\rbrace$, which exists because there are only finitely many such $x_n = y$ by $y$ not being an accumulation point. So, we can form a sequence $(f(y_k)) \subseteq f(X)$ converging to $0$. But, $0 \not\in f(X)$. So $f(X)$ is not closed, and by Heine-Borel, is not compact.


  \end{proof}
\end{enumerate}

\end{document}
