\documentclass[11pt, letterpaper]{article}
\usepackage{fullpage}
\usepackage{amsmath,amsthm,amsfonts,amssymb,amscd}
\usepackage{lastpage}
\usepackage{enumerate}
\usepackage{fancyhdr}
\usepackage{mathrsfs}
\usepackage{enumitem} % \setlist
% for \imageans: float for [H] so the figure floats
\usepackage{graphicx}
\usepackage{adjustbox}
\usepackage{float} 

\setlength{\parindent}{0.25in}
\setlength{\parskip}{0.05in}
% indent paragraphs in list
\setlist{  
  listparindent=\parindent,
  parsep=0pt,
}

% Include graphics in answer
\newcommand{\imageans}[1]
{%
    \begin{figure}[H]
        \centering
        \includegraphics[width=0.4\linewidth]{#1}
    \end{figure}
}

% comments inside align environnment
\newcommand{\comment}[1]{%
  \text{\phantom{(#1)}} \tag{#1}
}
\newtheorem{theorem}{Theorem}
\newtheorem{lemma}{Lemma}
% Cases for Proof environment
\newlist{pcases}{enumerate}{1}
\setlist[pcases]{
  label=\underline{Case~\arabic*}:\protect\thiscase.~,
  ref=\arabic*,
  align=left,
  labelsep=0pt,
  leftmargin=0pt,
  labelwidth=0pt,
  parsep=0pt
}
\newcommand{\case}[1][]{%
  \if\relax\detokenize{#1}\relax
    \def\thiscase{}%
  \else
    \def\thiscase{~#1}%
  \fi
  \item
}

% Edit these as appropriate
\newcommand\course{Math 3122}
\newcommand\hwtitle{HW 3}                  
\newcommand\name{David Tran}
\newcommand\studentid{251169871}

\fancypagestyle{firststyle}
{
    \headheight 35pt
    \lhead{\name}
    \lhead{\name\\\studentid}
    \chead{\textbf{\LARGE \hwtitle}}
    \rhead{\course \\ \today}
    \lfoot{}
    \cfoot{}
    \rfoot{\small\thepage}
    \headsep 1.5em
}

\DeclareUnicodeCharacter{2212}{-}
\begin{document}

\thispagestyle{firststyle}

\setlist[enumerate]{leftmargin=*} % remove enuemrate indentation

\begin{enumerate}
  \item First we show that $B(0, 2)$ can be covered by finitely many open balls.
  \begin{proof}
    Define 3 open balls in $\mathbb R$ on the standard metric as $B_1(0, 1)$, $B_2(-1, 1)$, $B_3(1, 1)$. Note that $B_1 = (-1, 1)$, $B_2 = (-2, 0)$, and $B_3 = (0, 2)$. Let $x \in B(0, 2)$. Then $x \in (-2, 2)$ so $x \in \bigcup_{k =1}^3 B_k = (-2, 2)$. That is, $B$ is covered by finitely many open balls.
  \end{proof}

  Next we show that this doesn't hold in $\ell_1$.
  \begin{proof}
    Consider the sequences $X = \lbrace (x_n)_k \mid k \in \mathbb N \rbrace$ defined such that $x_n = 3/2$ when $n = k$ and $x_n = 0$ otherwise. $X \subseteq \ell_1$ since only $x_k = 3/2$ is non-zero, so every series converges to $3/2$. Also, $X \subseteq B(0; 2)$, since for all $x \in X$, $d(x, 0) = 3/2 < 2$ for all $k \in \mathbb N$. So, any finite covering of $B(0; 2)$ must at least cover $X$. Note that any two distinct sequences $x, y \in X$ cannot be contained in a single open ball of radius 1. To see this, suppose for contradiction that $x, y \in B(z; 1)$ for some sequence $z$. Then $d(x, z), d(y, z) < 1$, so $d(x, z) + d(z, y) < 2$. But, $d(x, y) = 3$, so $d(x, y) > d(x, z) + d(z, y)$, violating the triangle inequality; a contradiction. Thus, every open ball of radius 1 can only contain one sequence in $X$. So, since $\vert X \vert = \vert \mathbb N \vert$, $X$ requires infinitely many open balls of radius 1 to be covered. So $B(0; 1)$ requires infinitely many such open balls to be covered.
  \end{proof}

  \item \begin{proof}
    Let $f: X \to X$ be a contraction on a metric space with metric $d$. Then there is an $\alpha \in \mathbb R$, $0 < \alpha < 1$ such that $d(f(x), f(y)) = \alpha d(x, y)$. Let $\epsilon > 0$. Fix $\delta = \epsilon/\alpha$. Then for all $x, y \in X$, if $d(x, y) < \delta$, then $d(f(x), f(y)) = \alpha d(x, y) < \alpha \cdot \frac{\epsilon}{\alpha} = \epsilon$. So $f$ is uniformly continuous.
  \end{proof}
\end{enumerate}

\end{document}
