\documentclass[11pt, letterpaper]{article}
\usepackage{fullpage}
\usepackage{amsmath,amsthm,amsfonts,amssymb,amscd}
\usepackage{lastpage}
\usepackage{enumerate}
\usepackage{fancyhdr}
\usepackage{mathrsfs}
\usepackage{enumitem} % \setlist
% for \imageans: float for [H] so the figure floats
\usepackage{graphicx}
\usepackage{adjustbox}
\usepackage{float} 

\setlength{\parindent}{0.25in}
\setlength{\parskip}{0.05in}
% indent paragraphs in list
\setlist{  
  listparindent=\parindent,
  parsep=0pt,
}

% Include graphics in answer
\newcommand{\imageans}[1]
{%
    \begin{figure}[H]
        \centering
        \includegraphics[width=0.4\linewidth]{#1}
    \end{figure}
}

% comments inside align environnment
\newcommand{\comment}[1]{%
  \text{\phantom{(#1)}} \tag{#1}
}
\newtheorem{theorem}{Theorem}
\newtheorem{lemma}{Lemma}
% Cases for Proof environment
\newlist{pcases}{enumerate}{1}
\setlist[pcases]{
  label=\underline{Case~\arabic*}:\protect\thiscase.~,
  ref=\arabic*,
  align=left,
  labelsep=0pt,
  leftmargin=0pt,
  labelwidth=0pt,
  parsep=0pt
}
\newcommand{\case}[1][]{%
  \if\relax\detokenize{#1}\relax
    \def\thiscase{}%
  \else
    \def\thiscase{~#1}%
  \fi
  \item
}

% Edit these as appropriate
\newcommand\course{Math 3122}
\newcommand\hwtitle{HW 4}                  
\newcommand\name{David Tran}
\newcommand\studentid{251169871}

\fancypagestyle{firststyle}
{
    \headheight 35pt
    \lhead{\name}
    \lhead{\name\\\studentid}
    \chead{\textbf{\LARGE \hwtitle}}
    \rhead{\course \\ \today}
    \lfoot{}
    \cfoot{}
    \rfoot{\small\thepage}
    \headsep 1.5em
}

\DeclareUnicodeCharacter{2212}{-}
\begin{document}

\thispagestyle{firststyle}

\setlist[enumerate]{leftmargin=*} % remove enuemrate indentation

\begin{enumerate}
  \item \begin{proof}
    Let $A$ and $B$ be connected and suppose for the sake of contradiction that $A \cup B$ is disconnected. Then there exists open, disjoint, non-empty subsets $U, V \subseteq A \cup B$ such that $A \cup B = U \cup V$. Choose $x \in A \cap B$. Then $x \in U$ or $x \in V$. Without loss of generality, suppose $x \in U$ and choose $y \in V$. Then $y \in A$ or $y \in B$. Without loss of generality, suppose $y \in A$. Since $x \in A \cap B$ and $x \in U$, then $x \in A \cap U$. Also, $y \in A \cap V$. We have that $A \subseteq A \cup B = U \cup V$. So neither $U$ nor $V$ are empty. Thus, since $A \subseteq A \cup B = U \cup V$, $U$ and $V$ are open, disjoint, non-empty subsets that cover $A$, contradicting the fact that $A$ is connected.
  \end{proof}
  \item \begin{proof}
    $(\rightarrow)$ Suppose $X$ is connected. Then the only subsets of $X$ that are clopen are $\emptyset$ and $X$. Since $f$ is a homeomorphism, it is continuous, so it preserves openess and closedness. Thus the only subsets of $f(X) = Y$ that are clopen are $f(\emptyset) = \emptyset$ and $Y$. So $Y$ is connected.

    $(\leftarrow)$ Since $f$ is a homeomorphism, it is bijective, so it has a continuous inverse. Then the same proof as above applies for $Y \to X$.
  \end{proof}
\end{enumerate}

\end{document}
