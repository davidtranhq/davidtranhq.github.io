\documentclass{beamer}
\usepackage{algorithmic}
\usetheme{Copenhagen}
\usepackage{tikz}
\usepackage{xcolor}

% Define a new color for fading
\definecolor{fade}{RGB}{200,200,200}

% Define a new command for fading text
\newcommand{\fadetext}[1]{\textcolor{fade}{#1}}

\AtBeginSection[]
{
  \begin{frame}
    \frametitle{Table of Contents}
    \tableofcontents[currentsection]
  \end{frame}
}
\beamerdefaultoverlayspecification{<+->}
%Information to be included in the title page:
\title{Parallel Algorithms for the Maximum Subarray Problem}
\author{David Tran}
\institute{University of Western Ontario}
\date{CS 4402, April 2024}

\begin{document}

\frame{\titlepage}

\section{Background}

\begin{frame}
  \frametitle{Maximum Subarray Problem}
  Let $A$ be a $d$-dimensional array of $n$ reals. Find a $d$-dimensional subarray $A'$ of $A$ such that the sum of the elements of $A'$ is maximized.
\end{frame}

\begin{frame}{Applications}
  \begin{itemize}
    \item Image processing: find the brightest region in an image
    \item Signal processing: find the loudest region in a sound wave
    \item Data mining: find the most profitable region in a dataset
    \item Bioinformatics: find the most conserved region in a DNA sequence
  \end{itemize}
\end{frame}

\begin{frame}{History (Serial version)}
  \begin{itemize}
    \item 2D version of the problem first proposed by Ulf Grenander in 1977 for maximum-likelihood estimation of patterns in images
    \item Grenander discovers $O(n^2)$ algorithm for 1D version
    \item Michael Shamos discovers $O(n \log n)$ algorithm overnight
    \item Jay Kadane invents $O(n)$ in less than a minute
  \end{itemize}
\end{frame}

\section{Main Results}

\begin{frame}{Main Result}
  \begin{alertblock}{Theorem}
    There exists parallel algorithms for $d = 1$ and $d = 2$ with $O(\log n)$ time complexity. The former is optimal on an EREW-PRAM, while the latter is optimal on a CREW-PRAM.
  \end{alertblock}
\end{frame}

\begin{frame}{Prefix/Suffix Sums and Maxima}
  % latex example of prefix/suffix sums in a beamer slide
  \begin{table}[h]
    \text{$Q = $}
    \begin{tabular}{|c|c|c|c|c|c|c|c|c|c|c|}
      \hline
      1 & 2 & 3 & 4 & 5 \\
      \hline
    \end{tabular}
  \end{table}

  \pause

  \begin{table}[h]
    \text{$P(Q) = $}
    \begin{tabular}{|c|c|c|c|c|c|c|c|c|c|c|}
      \hline
      1 & 3 & 6 & 10 & 15 \\
      \hline
    \end{tabular}
  \end{table}

  \pause

  \begin{table}[h]
    \text{$S(Q) = $}
    \begin{tabular}{|c|c|c|c|c|c|c|c|c|c|c|}
      \hline
      15 & 14 & 12 & 9 & 5 \\
      \hline
    \end{tabular}
  \end{table}

  \pause

  \begin{table}[h]
    \text{$M_P(Q) = $}
    \begin{tabular}{|c|c|c|c|c|c|c|c|c|c|c|}
      \hline
      1 & 2 & 3 & 4 & 5 \\
      \hline
    \end{tabular}
  \end{table}

  \pause

  \begin{table}[h]
    \text{$M_S(Q) = $}
    \begin{tabular}{|c|c|c|c|c|c|c|c|c|c|c|}
      \hline
      5 & 5 & 5 & 5 & 5 \\
      \hline
    \end{tabular}
  \end{table}
\end{frame}

\begin{frame}{Range Query in $O(1)$}
  The sum of elements in the subarray $[i, j]$ is $P[j] - P[i - 1]$.

  \begin{table}[h]
    \text{$Q = $}
    \begin{tabular}{|c|c|c|c|c|c|c|c|c|c|c|}
      \hline
      1 & 2 & 3 & 4 & 5 \\
      \hline
    \end{tabular}
  \end{table}
  
  \begin{table}[h]
    \text{$P(Q) = $}
    \begin{tabular}{|c|c|c|c|c|c|c|c|c|c|c|}
      \hline
      1 & 3 & 6 & 10 & 15 \\
      \hline
    \end{tabular}
  \end{table}

  Sum of $Q[1, 3] = P[3] - P[0] = 10 - 1 = 9$.
\end{frame}

\begin{frame}{Key Lemma}

\begin{alertblock}{Lemma 1}
  Let $Q$ be a 1D array of $n$ reals. Let $M_s^k$ be the maximum of the suffix sums of $q_0, \dots, q_k$ and $M_p^k$ the maximum of the prefix sums of $q_k, \dots, q_{n - 1}$. The maximum of the sums of all subarrays containing $q_k$ is

  $$
  Max(q_k) := M_s^k + M_p^k - q_k
  $$

  (the sum of the largest subarrays ending at $q_k$ and starting at $q_k$).
\end{alertblock}
\end{frame}

\begin{frame}{Proof (Sketch) of Lemma 1}

Suppose $[A, B]$ is satisfies the above Lemma. Then any other subarray containing $q_k$ (say $[a, b]$) has lesser sum.

\begin{table}[h]
  \begin{tabular}{|c|c|c|c|c|c|c|c|c|c|c|}
    \hline
    & $a$ & \dots & $A$ & \dots & $q_k$ & \dots & $b$ & \dots & $B$ &  \\
    \hline
  \end{tabular}
\end{table}

\end{frame}

\begin{frame}{Maximum Subarray Algorithm}
\begin{enumerate}
  \item Compute the prefix/suffix sums of $Q$ as $P$, $S$
  \item Compute the prefix maxima of $S$ as $M_p$ and the suffix maxima of $P$ as $M_s$
  \item For each $1 \leq k \leq n$, \begin{enumerate}
    \item $M_s^k = M_p[k] - S[k] + Q[k]$ (range query from $A$ to $k$)
    \item $M_p^k = M_s[k] - P[k] - Q[k]$ (range query from $k$ to $B$)
    \item $Max(q_k) = M_s^k + M_p^k - Q[k]$
  \end{enumerate}
  \item Return the maximum of $Max(q_k)$
\end{enumerate}
\end{frame}

\beamerdefaultoverlayspecification{<*>}

\begin{frame}[<*>]
  \frametitle{Maximum Subarray Algorithm (Parallel Version)}
  \begin{enumerate}
    \item Compute the prefix/suffix sums of $Q$ as $P$, $S$ \alert{(in parallel)}
    \item Compute the prefix maxima of $S$ as $M_p$ and the suffix maxima of $P$ as $M_s$ \alert{(in parallel)}
    \item For each $1 \leq k \leq n$, \alert{(in parallel)} \begin{enumerate}
      \item $M_s^k = M_p[k] - S[k] + Q[k]$ (range query from $A$ to $k$)
      \item $M_p^k = M_s[k] - P[k] - Q[k]$ (range query from $k$ to $B$)
      \item $Max(q_k) = M_s^k + M_p^k - Q[k]$
    \end{enumerate}
    \item Return the maximum of $Max(q_k)$ \alert{(in parallel)}
  \end{enumerate}
\end{frame}

\begin{frame}{Maximum Subarray Algorithm (Analysis)}
  Using $O(n /\log n)$ processors on EREW-PRAM:
  \begin{enumerate}
    \item \fadetext{Compute the prefix/suffix sums of $Q$ as $P$, $S$} \only<2->{$O(\log n)$}
    \item \fadetext{Compute the prefix maxima of $S$ as $M_p$ and the suffix maxima of $P$ as $M_s$} \only<3->{$O(\log n)$}
    \item \fadetext{For each $1 \leq k \leq n$,} \only<4->{$O(\log n)$} \begin{enumerate}
    \item \fadetext{$M_s^k = M_p[k] - S[k] + Q[k]$ (range query from $A$ to $k$)
    \item $M_p^k = M_s[k] - P[k] - Q[k]$ (range query from $k$ to $B$)
    \item $Max(q_k) = M_s^k + M_p^k - Q[k]$}
    \end{enumerate}
    \item \fadetext{Return the maximum of $Max(q_k)$} \only<5->{$O(\log n)$}
  \end{enumerate}

  \only<6->{$T(n)$: $O(\log n)$\\}
  \only<7->{$SU(n)$: $O(n)$ (Kadane)\\}
  \only<8->{Efficiency:$O(n)/O(\log n \cdot n/\log n) = O(1) (!!)$}
\end{frame}

\begin{frame}{Recall: Main Result}
  \begin{alertblock}{Theorem}
    There exists parallel algorithms for $d = 1$ and $d = 2$ with $O(\log n)$ time complexity. The former is optimal on an EREW-PRAM, while the latter is optimal on a CREW-PRAM.
  \end{alertblock}
\end{frame}

\beamerdefaultoverlayspecification{<+->}

\begin{frame}{Maximum Subarray Idea: $d = 2$}
% 5x5 matrix in latex with positive and negative entries
\begin{table}[htbp]
  \centering
  \begin{tabular}{|c|c|c|c|c|}
  \hline
  2 & -3 & 4 & -1 & 5 \\
  \hline
  -1 & 6 & -2 & 7 & -3 \\
  \hline
  4 & -2 & 8 & -4 & 9 \\
  \hline
  \end{tabular}
\end{table}

\begin{enumerate}
  \item Pick a 2D subarray of columns.
  \item Sum across the rows to form a 1D array.
  \item Apply the 1D algorithm to find the maximum subarray sum.
  \item Repeat for all subarrays of columns.
\end{enumerate}
\end{frame}

\begin{frame}{Maximum Subarray Algorithm: $d = 2$}
  
  \begin{enumerate}
    \item Replace each row by its prefix sums.
    \item Prepend a column of zeroes to the matrix.
    \item For each subarray of columns $C^{gh}$, $0 \leq g \leq h < n$ \begin{enumerate}
      \item Collapse (sum) the rows of $C^{gh}$ to form a a single column
      \item Apply the 1D algorithm to find the maximum subarray sum $M_{gh}$ for $C^{gh}$.
    \end{enumerate}
    \item Find the maximum of all $M_{gh}$ for $0 \leq g \leq h < n$.
  \end{enumerate}
\end{frame}

\beamerdefaultoverlayspecification{<*>}

\begin{frame}{Maximum Subarray Algorithm: $d = 2$ (Analysis)}
  \only<1->{
  Assume $n^3/\log n$ processors on a CREW-PRAM, and $n^3$ processors on an EREW-PRAM.
  }
  \begin{enumerate}
    \item \fadetext{Replace each row by its prefix sums.} \only<2->{$O(\log n)$}
    \item \fadetext{Prepend a column of zeroes to the matrix.} \only<3->{$O(\log n)$}
    \item \fadetext{For each subarray of columns $C^{gh}$, $0 \leq g \leq h < n$} \begin{enumerate}
      \item \fadetext{Collapse (sum) the rows of $C^{gh}$ to form a 1D array.} \only<4->{$O(\log n)$, using $n /\log n$ and $n$ processors for each $C^{gh}$ on CREW-PRAM and EREW-PRAM, respectively.}
      \item \fadetext{Apply the 1D algorithm to find the maximum subarray sum $M_{gh}$ for $C^{gh}$.} \only<5->{$O(\log n)$}
    \end{enumerate}
    \item \fadetext{Find the maximum of all $M_{gh}$ for $0 \leq g \leq h < n$.} \only<6->{$O(\log n)$}
  \end{enumerate}

  \only<7->{
  $T(n)$: $O(\log n)$ \\
  } \only<8-> {
  $SU(n)$: $O(n^3)$ \\
  } \only<9-> {
  Efficiency: $O(n^3)/O(\log n \cdot n^3/\log n) = O(1)$
  }
\end{frame}

\begin{frame}{Maximum Subarray Algorithm: $d > 2$}
  In general, for each of the $\binom{n}{2}^{d - 2}$ $d$-dimensional subarrays, we can collapse them to a $d - 1$-dimensional subarray and apply the $d - 1$ algorithm to it in $O(\log n)$ time.

  \begin{block}{Remark}
    The $d$-dimensional maximum subarray problem can be solved in $O(\log n)$ time with $n \cdot \binom{n}{2}^{d - 1}$ processors on a CREW-PRAM, and $n^2 \cdot \binom{n}{2}^{d - 1}$ processors on an EREW-PRAM.
  \end{block}
\end{frame}

\section{Recent Work}

\begin{frame}{Recent Work}
  \begin{itemize}
    \item (1998) A (very slightly) sub-cubic algorithm for the 2D serial version using matmul
    \item (2004) 2D parallel version of the algorithm designed for BSP/CGM models
    \item (2017) 2D parallel version of the algorithm (among other problems) with optimal communication complexity on the systolic array model (FGPA, ASIC)
  \end{itemize}
\end{frame}

\begin{frame}
\frametitle{Citations}
\begin{itemize}
  \item Perumalla, K., \& Deo, N. (1995). \textit{Parallel algorithms for maximum subsequence and maximum subarray}. Parallel Processing Letters. World Scientific Publishing Company.
  \item Alves, C.E.R., Cáceres, E.N., Song, S.W. (2004). BSP/CGM Algorithms for Maximum Subsequence and Maximum Subarray.
  \item Bae, S.E.; Shinn, T.-W.; Takaoka, T. Efficient Algorithms for the Maximum Sum Problems. Algorithms (2017), 10, 5. https://doi.org/10.3390/a10010005
\end{itemize}
\end{frame}



\end{document}