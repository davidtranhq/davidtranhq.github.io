\documentclass[11pt, letterpaper]{article}
\usepackage{fullpage}
\usepackage{amsmath,amsthm,amsfonts,amssymb,amscd}
\usepackage{lastpage}
\usepackage{enumerate}
\usepackage{fancyhdr}
\usepackage{mathrsfs}
\usepackage{enumitem} % \setlist
% for \imageans: float for [H] so the figure floats
\usepackage{graphicx}
\usepackage{adjustbox}
\usepackage{float}
\usepackage{listings}
\usepackage{xcolor}

\lstdefinestyle{verbatimStyle}{
    basicstyle=\ttfamily,
    columns=fullflexible,
    keepspaces=true,
    keywordstyle=\color{blue},
    commentstyle=\color{green!40!black},
    stringstyle=\color{orange},
    numbers=left,
    numberstyle=\tiny,
    % numbersep=5pt,
    breaklines=true,
    breakatwhitespace=true,
    tabsize=4,
}

\setlength{\parindent}{0.25in}
\setlength{\parskip}{0.05in}
% indent paragraphs in list
\setlist{  
  listparindent=\parindent,
  parsep=0pt,
}

% Include graphics in answer
\newcommand{\imageans}[1]
{%
    \begin{figure}[H]
        \centering
        \includegraphics[width=0.4\linewidth]{#1}
    \end{figure}
}

% comments inside align environnment
\newcommand{\comment}[1]{%
  \text{\phantom{(#1)}} \tag{#1}
}
\newtheorem{theorem}{Theorem}
\newtheorem{lemma}{Lemma}
% Cases for Proof environment
\newlist{pcases}{enumerate}{1}
\setlist[pcases]{
  label=\underline{Case~\arabic*}:\protect\thiscase.~,
  ref=\arabic*,
  align=left,
  labelsep=0pt,
  leftmargin=0pt,
  labelwidth=0pt,
  parsep=0pt
}
\newcommand{\case}[1][]{%
  \if\relax\detokenize{#1}\relax
    \def\thiscase{}%
  \else
    \def\thiscase{~#1}%
  \fi
  \item
}

% Edit these as appropriate
\newcommand\course{CS 4402}
\newcommand\hwtitle{HW 1}                  
\newcommand\name{David Tran}
\newcommand\studentid{251169871}

\fancypagestyle{firststyle}
{
    \headheight 35pt
    \lhead{\name}
    \lhead{\name\\\studentid}
    \chead{\textbf{\LARGE \hwtitle}}
    \rhead{\course \\ \today}
    \lfoot{}
    \cfoot{}
    \rfoot{\small\thepage}
    \headsep 1.5em
}

\DeclareUnicodeCharacter{2212}{-}
\begin{document}

\thispagestyle{firststyle}

\setlist[enumerate]{leftmargin=*} % remove enuemrate indentation

\begin{enumerate}
  \item Let $A$ be a lower-triangular $n \times n$ matrix. The following computes and returns its inverse.
  \begin{verbatim}
    fn lower_triangular_inverse(row, col, n, A)
    {
        A1_inv = spawn lower_triangular_inverse(row, col, n/2);
        A3_inv = lower_triangular_inverse(row + n/2, col + n/2, n/2);
        sync;
        A2 = A.submatrix(row + n/2, col, n/2); 
        result = matrix with (A1_inv) starting at (0,0),
            (-A3_inv * A2 * A1_inv) starting at (n/2, 0),
            (A3_inv) starting at (n/2, n/2)
        return result;
    } 
  \end{verbatim}

  \item The main algorithm is shown below.
  \lstinputlisting[language=c++,style=verbatimStyle]{code/lower_triangular_inverse.cpp}
  Note that \verb|SquareMatrixView::submat| runs in constant time, matrix multiplication runs in $\Theta(n^3)$, and \verb|Matrix::write_submat| runs in $\Theta(n)$. The recurrence is

  $$
  T(n) = 2T(n/4) + c(n/4)^3
  $$
\end{enumerate}
\end{document}
