\documentclass[11pt, letterpaper]{article}
\usepackage{fullpage}
\usepackage{amsmath,amsthm,amsfonts,amssymb,amscd}
\usepackage{lastpage}
\usepackage{enumerate}
\usepackage{fancyhdr}
\usepackage{mathrsfs}
\usepackage{enumitem} % \setlist
% for \imageans: float for [H] so the figure floats
\usepackage{graphicx}
\usepackage{adjustbox}
\usepackage{float} 

\setlength{\parindent}{0.25in}
\setlength{\parskip}{0.05in}
% indent paragraphs in list
\setlist{  
  listparindent=\parindent,
  parsep=0pt,
}

% Include graphics in answer
\newcommand{\imageans}[1]
{%
    \begin{figure}[H]
        \centering
        \includegraphics[width=0.4\linewidth]{#1}
    \end{figure}
}

% comments inside align environnment
\newcommand{\comment}[1]{%
  \text{\phantom{(#1)}} \tag{#1}
}
\newtheorem{theorem}{Theorem}
\newtheorem{lemma}{Lemma}
% Cases for Proof environment
\newlist{pcases}{enumerate}{1}
\setlist[pcases]{
  label=\underline{Case~\arabic*}:\protect\thiscase.~,
  ref=\arabic*,
  align=left,
  labelsep=0pt,
  leftmargin=0pt,
  labelwidth=0pt,
  parsep=0pt
}
\newcommand{\case}[1][]{%
  \if\relax\detokenize{#1}\relax
    \def\thiscase{}%
  \else
    \def\thiscase{~#1}%
  \fi
  \item
}

% Edit these as appropriate
\newcommand\course{Math 3124}
\newcommand\hwtitle{HW 7}                  
\newcommand\name{David Tran}
\newcommand\studentid{251169871}

\fancypagestyle{firststyle}
{
    \headheight 35pt
    \lhead{\name}
    \lhead{\name\\\studentid}
    \chead{\textbf{\LARGE \hwtitle}}
    \rhead{\course \\ \today}
    \lfoot{}
    \cfoot{}
    \rfoot{\small\thepage}
    \headsep 1.5em
}

\DeclareUnicodeCharacter{2212}{-}
\begin{document}

\thispagestyle{firststyle}

\setlist[enumerate]{leftmargin=*} % remove enuemrate indentation

\begin{enumerate}
  \item \begin{proof}
    Let $f$ be analytic and non-constant on the closure of a bounded region $D$. Suppose for contradiction that $\operatorname{Re} f$ is maximum at an interior point $z_0 \in D$. By the open mapping theorem, $f(D)$ is open, so $f(z_0) + \epsilon \in f(D)$ for some positive real $\epsilon$. But then $\operatorname{Re}(f(z_0) + \epsilon) > \operatorname{Re}(f(z_0))$, contradicting that $\operatorname{Re} f$ is maximum at $z_0$.
    
    Similarily, if $\operatorname{Re} f$ is minimum at an interior point $z_0 \in D$, then $f(z_0) - \epsilon \in f(D)$ for some positive real $\epsilon$, and $\operatorname{Re}(f(z_0) - \epsilon) < \operatorname{Re}(f(z_0))$, contradicting that $\operatorname{Re} f$ is minimum at $z_0$.

    Analogously, $\operatorname{Im} f$ can't be maximum or minimum at an interior point $z_0 \in D$, for then $f(z_0) \pm \epsilon \in f(D)$ for some imaginary $\epsilon$ with positive modulus would be smaller than the supposed minimum or larger than the supposed maximum.

    Thus, $\operatorname{Re} f$ and $\operatorname{Im} f$ must both attain their maximum and minimum on the boundary of $D$.
  \end{proof}

  \item First we prove the following lemma.
  
  \begin{lemma}
    Let $f: S \to T$ be a non-constant analytic function on its domain. If $f(z) \in T$ is a boundary point of $T$, then $z$ is a boundary point of $S$.
  \end{lemma}
  \begin{proof}
    Suppose otherwise: that $f(z)$ is a boundary point of $T$ but $z$ not a boundary point of $S$. Then $z$ is an interior point of $S$, so there exists a disk $D(z; r) \subset S$. By the open mapping theorem, $f(D(z; r))$ is open, so $f(z)$ is an interior point of $T$, contradicting that $f(z)$ is a boundary point of $T$.
  \end{proof}
  
  Now we prove the theorem.

  \begin{proof}
    We have that $B_\alpha: D(0; 1) \to A$ is analytic in its domain. We show that $A$ is the unit disk, and $B_\alpha$ is a bijection. First, note that if $\vert z \vert = 1$, then $\vert B_\alpha(z) \vert = 1$. So,by the Maximum-Modulus theorem, since $B_\alpha$ is non-constant, (e.g. $B_\alpha(0) = -\alpha \neq 1$), there is some $\vert B_\alpha(z) \vert < 1$ for some $z \in \operatorname{Int} D(0; 1)$. Let $\alpha \in A$ be one such value, with $\vert \alpha \vert < 1$. Consider an arbitrary chord $T$ of $C(0; 1)$ passing through $\alpha$. Let $X = T \setminus C(0; 1)$ (that is, $T$ excluding its endpoints). Then, $X \subseteq A$, since if it weren't, then there would be some $f(z) \in X$ that is a boundary point of $A$. But, this isn't possible, since by the Lemma, $z$ would be a boundary point with $\vert z \vert = 1$. But then $\vert f(z) \vert = 1$, contradicting that $f(z)$ is on a chord in the interior of $D(0; 1)$. So $X \subseteq A$. Thus, $A$ contains all chords (minus their endpoints) passing through $\alpha$, so $A$ is the interior of the unit disk. Finally, since $f$ is continuous and $D(0; 1)$ is compact, $A$ is compact. So $A$ is the unit disk.

    Now we show that $B_\alpha$ is a bijection by showing that it, composed with its inverse, is an identity map in $D(0; 1)$. Define its inverse $B_\alpha^{-1}: A \to D(0; 1)$ by

    \begin{align*}
      B_\alpha^{-1}(\beta) = \frac{\beta + \alpha}{1 + \overline{\alpha}\beta}
    \end{align*}

    Then \begin{align*}
      (B_\alpha^{-1} \circ B_\alpha)(z)
      &= \frac{\frac{z - \alpha}{1 - \overline{\alpha}z} + \alpha}
      {1 + \overline{\alpha}\frac{z + \alpha}{1 + \overline{\alpha}z}} \\
      &= \frac{z - \alpha \bar \alpha z}{1 - \alpha \bar \alpha} \\
      &= z
    \end{align*}

    Finally, note that this inverse is analytic on the unit disk, since it is a rational functions whose denominator is non-zero within the unit disk, since $1 + \overline \alpha z = \overline \alpha(1/\overline \alpha + z)$, and $\vert 1/\overline \alpha \vert > 1$ since $\vert \alpha \vert < 1$, and $\vert z \vert \leq 1$.
  \end{proof}

  \item \begin{proof}
    First, note that $f$ has finitely many zeroes inside the unit disk, for if it didn't, then by compactness of the unit disk, there would be a sequence of zeroes convergent to a point within the domain of analycity of $f$. Then by the Uniqueness Theorem, $f \equiv 0$; contradicting that $\vert f \vert = 1$ on $\vert z \vert = 1$. So let $\alpha_1, \dots, \alpha_n$ be the finitely many zeros of $f$. Then 

    $$
    g(z) = \frac{f(z)}{\prod_{j = 1}^n \frac{z - \alpha_j}{1 - \overline{\alpha_j}z}}
    $$

    is non-zero at all points inside the unit disk, and $\vert f \vert = 1$ on the unit disk boundary. So by the Maximum-Modulus and Minimum-Modulus theorems, $g$ is constant. So

    $$
    f(z) = C \prod_{j = 1}^n \frac{z - \alpha_j}{1 - \overline{\alpha_j}z}
    $$

    Finally, since $f$ is entire, $\alpha_1 = \dots = \alpha_n = 0$, so $f(z) = Cz^n$.
  \end{proof}

  \item Let $\alpha_1, \dots, \alpha_n$ be the zeroes of $Q$. Define
  
  $$
  g(z) = f(z)\prod_{j = 1}^n B_{\alpha_j}(z)
  $$

  then $B_{\alpha_j}(z) = 0$ when $z = \alpha_1, \dots, \alpha_n$, so $g$ has no poles within the unit disk. And since $\vert B_\alpha(z) \vert = 1$ when $\vert z \vert = 1$ (from above), $\vert f(z) \vert = \vert g(z) \vert$ when $\vert z \vert = 1$.

  \item Let $g(z) = \frac{1}{10} f(2z)$ so that the image of $g$ is contained in the unit disk and $g(1/2) = 0$.
  $$
  g(z) \ll B_{1/2}(z) = \frac{z - \frac{1}{2}}{1 - \frac{1}{2}z}
  $$
  So $g(1/4) \ll 2/7$. So $f(1/2) \ll 20/7$. This upper bound is attained by $f(z) = 10B_{1/2}(z/2)$ since $f(1/2) = 10B_{1/2}(1/4) = 20/7$.

  \item \begin{proof}
    We prove the contrapositive. Suppose that a region $D$ is not simply connected. Then there is a point in its complement $z \in \tilde D$ such that every path connecting $z$ to $\infty$ has some point on the path $\gamma(t)$ with $d(\gamma(t), \tilde D) > \epsilon$ for some $\epsilon > 0$. Consider the straight line paths connecting $z$ to $\infty$, $\alpha(t) = z + t$ and $\beta(t) = z - t$. Choose $t_\alpha$ and $t_\beta$ such that $\alpha(t_\alpha) = a$ and $\beta(t_\beta) = b$ are epsilon away from $\tilde D$. Then $a, b \in D$, but the straight line path $L$ connecting $a$ and $b$ is not entirely contained within $D$, since $z \in L$. So $D$ is not convex.
  \end{proof}
\end{enumerate}
\end{document}
