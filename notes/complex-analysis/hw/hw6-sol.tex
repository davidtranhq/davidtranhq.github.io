\documentclass[11pt, letterpaper]{article}
\usepackage{fullpage}
\usepackage{amsmath,amsthm,amsfonts,amssymb,amscd}
\usepackage{lastpage}
\usepackage{enumerate}
\usepackage{fancyhdr}
\usepackage{mathrsfs}
\usepackage{enumitem} % \setlist
% for \imageans: float for [H] so the figure floats
\usepackage{graphicx}
\usepackage{adjustbox}
\usepackage{float} 

\setlength{\parindent}{0.25in}
\setlength{\parskip}{0.05in}
% indent paragraphs in list
\setlist{  
  listparindent=\parindent,
  parsep=0pt,
}

% Include graphics in answer
\newcommand{\imageans}[1]
{%
    \begin{figure}[H]
        \centering
        \includegraphics[width=0.4\linewidth]{#1}
    \end{figure}
}

% comments inside align environnment
\newcommand{\comment}[1]{%
  \text{\phantom{(#1)}} \tag{#1}
}
\newtheorem{theorem}{Theorem}
\newtheorem{lemma}{Lemma}
% Cases for Proof environment
\newlist{pcases}{enumerate}{1}
\setlist[pcases]{
  label=\underline{Case~\arabic*}:\protect\thiscase.~,
  ref=\arabic*,
  align=left,
  labelsep=0pt,
  leftmargin=0pt,
  labelwidth=0pt,
  parsep=0pt
}
\newcommand{\case}[1][]{%
  \if\relax\detokenize{#1}\relax
    \def\thiscase{}%
  \else
    \def\thiscase{~#1}%
  \fi
  \item
}

% Edit these as appropriate
\newcommand\course{Math 3124}
\newcommand\hwtitle{HW 6}                  
\newcommand\name{David Tran}
\newcommand\studentid{251169871}

\fancypagestyle{firststyle}
{
    \headheight 35pt
    \lhead{\name}
    \lhead{\name\\\studentid}
    \chead{\textbf{\LARGE \hwtitle}}
    \rhead{\course \\ \today}
    \lfoot{}
    \cfoot{}
    \rfoot{\small\thepage}
    \headsep 1.5em
}

\DeclareUnicodeCharacter{2212}{-}
\begin{document}

\thispagestyle{firststyle}

\setlist[enumerate]{leftmargin=*} % remove enuemrate indentation

\begin{enumerate}
  \item Using the fact that $\sum_{k = 0}^\infty (-1)^n r^k = \frac{1}{1 + r}$ when $\vert r \vert < 1$,
  \begin{align*}
    \frac{1}{z} &= \frac{1}{1 + i + (z - 1 - i)} \\
    &= \frac{1}{(1 + i)(1 + \frac{z - 1 - i}{1 + i})} \\
    &= \frac{1}{1 + i} \sum_{n = 0}^\infty (-1)^n \left(\frac{z - (1 + i)}{1 + i}\right)^n \\
  \end{align*}
  when $\vert z \vert < 1 + i$.
  \item Factoring and applying partial fraction decomposition,
  \begin{align*}
    \frac{1}{1 - z - 2z^2} = \frac{1}{-(2z - 1)(z + 1)} = - \frac{A}{2z - 1} - \frac{B}{z + 1}
  \end{align*}
  and solving for $A, B$ in $-A(z + 1) - B(2z - 1) = 1$ for all $z$ gives $A = -2/3$ and $B = 1/3$.
  So, \begin{align*}
    f(z) &= \frac{2}{3} \left(\frac{1}{1 - 2z}\right) + \frac{1}{3} \left(\frac{1}{1 + z}\right) \\
    &= \frac{2}{3} \sum_{n = 0}^\infty (2z)^n + \frac{1}{3} \sum_{n = 0}^\infty (-1)^n z^n \\
    &= \sum_{n = 0}^\infty \frac{2^{n + 1} + (-1)^n}{3} z^n
  \end{align*}
  when $\vert z \vert < 1/2$.

  \item \begin{proof}
    Suppose for contradiction that for some function $f$ analytic in $\vert z \vert \leq 1$,
    $f(\frac{1}{n}) = \frac{1}{n + 1}$ for all $n \in \mathbb{N}$. Consider the sequence $z_n = \frac{1}{n}$, $n \in \mathbb N$.
    Then $z_n \to 0$ as $n \to \infty$, so by the Uniqueness Theorem, $f(z) = f(z_n) = \frac{z_n}{1 + z_n}$ for $\vert z \vert \leq 1$.
    But then, $f$ is discontinuous at $z = -1$, contradicting the fact that it is analyatic in $\vert z \vert \leq 1$.
  \end{proof}

  \item \begin{proof}
    Let $z_n = \frac{1}{n}$ for $n \in \mathbb N$. Then $z_n \subseteq \mathbb R$, so for some fixed $x \in \mathbb R$,
    $f(z) = \sin(x + z)$ and  $g(z) = \sin x \cos z + \cos x \sin z$ coincide for all $z_n$ (by the trigonometric identity for the reals). Since $\sin$ and $\cos$ are entire and $z_n \to 0$ as $n \to \infty$, by the Uniqueness theorem, $f(z) = g(z)$ for all $z \in \mathbb C$. Now consider $f^\star(z_1, z_2) = \sin(z_1 + z_2)$ and $g^\star(z_1, z_2) = \sin z_1 \cos z_2 + \cos z_1 \sin z_2$. Since $f(z) = g(z)$, $f^\star(z_1, z_2) = g^\star(z_1, z_2)$ for all $z_1 \subseteq (z_n)$ and $z_2 \in \mathbb C$. Then because $z_n$ converges to 0 and $f^\star$ and $g^\star$ are entire, $f^\star(z_1, z_2) = g^\star(z_1, z_2)$ for all $z_1, z_2 \in \mathbb C$.
  \end{proof}

  \item Let $z = x + iy$. Then
  \begin{align*}
    \vert z^2 - z \vert^2
    &= \vert z(z - 1) \vert^2 \\
    &= \vert (x + iy)(x + iy - 1) \vert^2 \\
    &= (x^2 - x - y^2)^2 + (2xy - y)^2 \\
  \end{align*}
  By the Maximum-Modulus theorem, the maximum occurs on the boundary of the disk, that is, when $\vert z \vert = 1$. Constraining to $x^2 + y^2 = 1$ by substituting $y^2 = 1 - x^2$ above, we have
  \begin{align*}
    \vert z^2 - z \vert^2 &= (x^2 - x - (1 - x^2))^2 + (2x(\sqrt{1 - x^2}) - \sqrt{1 - x^2})^2 \\
    &= 2 - 2x
  \end{align*}

  So the modulus is monotonically decreasing with respect to $x$. Since the domain is $\vert z \vert \leq 1$, the maximum occurs at $x = -1$, that is $z = -1$.

  (The following was added after the due date. If the late penalty is too high, please ignore this part.)
  The minimum modulus occurs at $z = 0$ at $z = 1$, where $\vert z^2 - z \vert = 0$.
  (end of late addition)

  \item \begin{proof}
    By the Cauchy Integral Formula, \begin{align*}
      \vert f(z_0)^n \vert 
      &= \left\vert \frac{1}{2\pi i} \int_C \frac{f(z)^n}{z - z_0} dz \right\vert \\
      &\leq \frac{1}{2\pi} \cdot \frac {M^n}{K'} \cdot 2 \pi R \\
      &= KM^n
    \end{align*}
    by the M-L formula, where $K = R/K'$ for some constant $K'$ dependent on the distance between $z_0$ and $C$. Note that $K'$ is constant and bounded since $z_0$ is inside $C$. Taking the $n$-th root of both sides, we have $\vert f(z_0) \vert \leq K^{\frac{1}{n}} M$, so as $n \to \infty$, $\vert f(z_0) \vert \leq M$.
  \end{proof}
\end{enumerate}

\end{document}
