\documentclass[11pt, letterpaper]{article}
\usepackage{fullpage}
\usepackage{amsmath,amsthm,amsfonts,amssymb,amscd}
\usepackage{lastpage}
\usepackage{enumerate}
\usepackage{fancyhdr}
\usepackage{mathrsfs}
\usepackage{enumitem} % \setlist
% for \imageans: float for [H] so the figure floats
\usepackage{graphicx}
\usepackage{adjustbox}
\usepackage{float} 

\setlength{\parindent}{0.25in}
\setlength{\parskip}{0.05in}
% indent paragraphs in list
\setlist{  
  listparindent=\parindent,
  parsep=0pt,
}

% Include graphics in answer
\newcommand{\imageans}[1]
{%
    \begin{figure}[H]
        \centering
        \includegraphics[width=0.4\linewidth]{#1}
    \end{figure}
}

% comments inside align environnment
\newcommand{\comment}[1]{%
  \text{\phantom{(#1)}} \tag{#1}
}
\newtheorem{theorem}{Theorem}
\newtheorem{lemma}{Lemma}
% Cases for Proof environment
\newlist{pcases}{enumerate}{1}
\setlist[pcases]{
  label=\underline{Case~\arabic*}:\protect\thiscase.~,
  ref=\arabic*,
  align=left,
  labelsep=0pt,
  leftmargin=0pt,
  labelwidth=0pt,
  parsep=0pt
}
\newcommand{\case}[1][]{%
  \if\relax\detokenize{#1}\relax
    \def\thiscase{}%
  \else
    \def\thiscase{~#1}%
  \fi
  \item
}

% Edit these as appropriate
\newcommand\course{Math 3120}
\newcommand\hwtitle{HW 2}                  
\newcommand\name{David Tran}
\newcommand\studentid{251169871}

\fancypagestyle{firststyle}
{
    \headheight 35pt
    \lhead{\name}
    \lhead{\name\\\studentid}
    \chead{\textbf{\LARGE \hwtitle}}
    \rhead{\course \\ \today}
    \lfoot{}
    \cfoot{}
    \rfoot{\small\thepage}
    \headsep 1.5em
}

\DeclareUnicodeCharacter{2212}{-}
\begin{document}

\thispagestyle{firststyle}

\setlist[enumerate]{leftmargin=*} % remove enuemrate indentation

\begin{enumerate}
  \item \begin{enumerate}
    \item \begin{proof}
      Suppose it were. Then let $\langle A \rangle = \mathbb Q$ with $A \subseteq \mathbb Q$ where $A$ is finite. Let $d$ be the product of every denominator in $A$. Then any linear combination of the rationals in $a$ can be written as $x/d$ for some term $x$. Now choose a prime $p$ that does not divide $d$. Then $1/p \in \mathbb Q$ can not be generated by $A$, since $x/d$ is reducible to $1/p$ if and only if $p$ divides $d$, which it doesn't by construction; a contradiction.
    \end{proof}
    \item \begin{proof} Consider $A := \lbrace \frac{a}{2^n} : n \in \mathbb N, a \in \mathbb Z \rbrace$. It is a subgroup of $\mathbb Q$ since $A \subseteq \mathbb Q$ and $a/2^n - b/2^n = (a - b)/2^n \in A$, and it is proper since $1/3 \in \mathbb Q - A$. It is not cyclic since if it were generated by a single rational, say $r \in A$, then $r/2 \in A$ but $ar \neq r/2$ for $a \in \mathbb Z$.
    \end{proof}
    \item \begin{proof}
      Let $a/b \in \mathbb Q^+$. By closure, it suffices to show that $a$ and $1/b$ are generated by the set of interest $A$. Let $a = p_1^{e_1}\dots p_n^{e_n}$ be the prime factorization of $a$. Then $a = (p_1^{-1})^{-e_1} \dots (p_n^{-1})^{-e_n}$, so $a \in \langle A \rangle$. Similarily, let $b = q_1^{f_1} \dots q_m^{f_m}$ be the prime factorization of $b$, then $1/b = ((q_1)^{-1})^{-f_1}\dots (q_m^{-1})^{-f_m}$, so $1/b \in \langle A \rangle$. 
    \end{proof}
  \end{enumerate}
  \item \begin{enumerate}
    \item The centralizers for $\langle i \rangle, \langle j \rangle$, and $\langle k \rangle$ are at least the subgroups themselves, since each are cyclic and thus Abelian. The only greater subgroup is $Q_8$ itself, which is not Abelian and thus cannot be part of the centralizer. So the centralizers for $\langle i \rangle, \langle j \rangle, \langle k \rangle$ are the subgroups themselves. The centralizer for $\langle 1 \rangle$ and $\langle -1 \rangle$ is $Q_8$, since $1$ and $-1$ commute with all of $Q_8$. Finally, the centralizer for $Q_8$ is $\lbrace 1, -1 \rbrace$, since neither $i$, $j$, nor $k$ commutes with $j$, $k$, nor $i$, respectively and ny the lattice diagram, there are no greater subgroups for $Q_8$ that could be the centralizer.
    
    Since the centralizer is a subgroup of the normalizer, the normalizer of $\langle 1 \rangle$ and $\langle -1 \rangle$ is $Q_8$. For the normalizer of $\langle i \rangle$, note that $j \langle i \rangle j^{-1} = \lbrace 1, i, -1, -i \rbrace = \langle i \rangle$. So, $j \in N_{Q_8}( \langle i \rangle)$, and since $C_{Q_8}(\langle i \rangle) = \langle i \rangle$, the normalizer must be $Q_8$. Symmetrically, the normalizer for $\langle j \rangle$ and $\langle k \rangle$ is $Q_8$. Finally, every group is its own normalizer, so $N_{Q_8}(Q_8) = Q_8$.
    
    \item From above, since the normalizer of every subgroup is $Q_8$, every subgroup is normal.
    
    For the isomorphism type of the quotient of each subgroup: we have $\vert Q_8 : \langle -1 \rangle \vert = 4$, so $Q_8/\langle -1 \rangle$ is congruent to either $V_4$ or $\mathbb Z_4$. But, for any $g \in Q_8/\langle -1 \rangle$, $g^2\langle -1 \rangle = \langle -1 \rangle$, so the elements $i\langle -1 \rangle, j \langle -1 \rangle, k \langle -1 \rangle$ have order 2, so $Q_8/\langle -1 \rangle$ cannot be congruent to $\mathbb Z_4$. Thus $Q_8/\langle -1 \rangle \cong V_4$.

    Next, we have $\vert Q_8 : \langle i \rangle \vert = 2$. Thus $Q/\langle i \rangle \cong \mathbb Z_2$. Similarily, $Q/\langle j \rangle \cong Q/\langle k \rangle \cong \mathbb Z_2$.
  \end{enumerate}
  \item \begin{enumerate}
    \item \begin{proof}
      Let $A$ be divisible with $B \leq A$. Let $aB \in A/B$ with $a \in A$. Since $x^n = a$ for some non-zero integer $n$ and $x \in A$, $aB = (x^n)B = (xB)^n$. So $A/B$ is divisible.
    \end{proof}
    \item \begin{itemize}
      \item Finite Abelian groups are not necessarily divisible: consider $\mathbb Z_2$ under addition. There is no element $y \in \mathbb Z_2$ such that $2y = 1$, since $2(0) = 2(1) = 0$.
      \item $\mathbb Z$ is not divisible: there is no element $y \in \mathbb Z$ such that $2y = 1$ since the product of an even integer with any integer is even.
      \item $\mathbb Q$ is divisible since for any $r \in \mathbb Q$ and $n \in \mathbb N$, $r/n \in \mathbb Q$ and $(r/n)n = r$.
      \item $\mathbb Q/\mathbb Z$ is divisible: let $r\mathbb Z \in \mathbb Q/\mathbb Z$ and $n \in \mathbb N$. Then $(r/n)\mathbb Z \in \mathbb Q/\mathbb Z$ and $(r/n)\mathbb Z \cdot n\mathbb Z = r \mathbb Z$.
    \end{itemize}
  \end{enumerate}

  \item \begin{enumerate}
    \item \begin{proof}
      If $A$ is Abelian, then $A \times A$ is Abelian since $(x, y)(u, v) = (xu, yv) = (ux, vy) = (u, v)(x, y)$ for $x, y, u, v \in A$. The subgroup of any Abelian group is normal, so $D \leq A \times A$ is normal.
    \end{proof}
    \item \begin{proof}
      Let $\sigma = (1 \enspace 2 \enspace 3)$, $\tau = (1 \enspace 2)$, and $\rho = (2 \enspace 3)$, with $\sigma, \tau, \rho \in S_3$. Then $(\rho, \rho) \in D$ and $(\sigma, \tau) \in S_3 \times S_3$, but
      \begin{align*}
        (\sigma, \tau)^{-1}(\rho, \rho)(\sigma, \tau)
        = (\tau, \sigma^{-1}) \not\in D.
      \end{align*}
      so $D$ is not normal.
    \end{proof}
  \end{enumerate}

  \item \begin{enumerate}
    \item \begin{proof}
      First we show that for any $g \in G$, $g = x^nz$ where $z \in Z(G), n \in \mathbb Z$, and $x$ is such that $G/Z(G) = \langle xZ(G) \rangle$. Let $g \in G$. Then $gZ(G) = x^n Z(G)$ for some $m$, so $x^{-n}g \in Z(G)$. So there is some $z \in Z(G)$ with $x^{-n}g = z$, that is, $g = x^nz$.

      Now let $g, h \in G$. From above,
      $gh = (x^n z_1) (x^m z_2)
        = (x^m z_2) (x^nz_1)
        = hg$,
      using commutativity from the fact that $z_1, z_2 \in Z(G)$.
    \end{proof}
    \item \begin{proof}
      By Lagrange's theorem, the order of $Z(G)$ is either $pq, p, q$ or 1. If the order is $pq$, then $Z(G) = G$ so $G$ is Abelian. If it is $p$ (or $q$), then $\vert G/Z(G) \vert = q$ (or $p$), so it is cyclic and thus Abelian by (a). To see that $G/Z(G)$ is cyclic, note that for some non-identity element $x \in Z(G)$, $\vert x \vert = q$ (or $p$) by Lagrange's theorem and the fact that $q$ (or $p$) is prime. So, $\vert \langle x \rangle \vert = \vert Z(G) \vert$, so $\langle x \rangle = Z(G)$.

      Thus, $G$ is either Abelian or $\vert Z(G) \vert = 1$ so $Z(G) = 1$.
      \end{proof}
  \end{enumerate}
  \item \begin{proof}
    Since $xH = Hy$ and $1 \in H$, choose $h \in H$ such that $x(1) = x = hy$. Thus $xy^{-1} = h \in H$, so $xy^{-1} \in H$, which implies $Hx = Hy$. Hence $xH = Hy = Hx$. Finally, since $xH = Hx$, $xHx^{-1} = H$, so $x \in N(H)$.
  \end{proof}
  \item \begin{proof}
    Since $G$ is finite and $H, N \leq G$, $\vert HN \vert = \frac{\vert H \vert \vert N \vert}{\vert H \cap N \vert}$.
    Also, from Corollary 15 in Chapter 3, since $N \trianglelefteq G$, $HN \leq G$.
    And, $\vert G : N \vert = \vert G \vert/\vert N \vert$ implies $\vert N \vert = \vert G \vert / \vert G : N \vert$. So,

    \begin{align*}
      \vert G \vert &= \frac{\vert H \vert \vert N \vert}{\vert H \cap N \vert} \vert G : HN \vert \\
      &= \frac{\vert H \vert \vert G \vert}{\vert H \cap N \vert \vert G : N \vert} \vert G : HN \vert
    \end{align*}
    Dividing on both sides by $\vert G \vert$ gives $\vert H \vert \vert G : HN \vert = \vert H \cap N \vert \vert G : N \vert$. Finally, since $(\vert H \vert, \vert G : N \vert) = 1$, we must have $\vert H \vert = \vert H \cap N \vert$ and $\vert G : HN \vert = \vert G : N \vert$. So $H \leq N$.
  \end{proof}

  \item \begin{proof}
    Since $\vert H \vert \vert G : M \vert = p$, either $(\vert H \vert, \vert G : M \vert) = 1$, or $(\vert H \vert, \vert G : M \vert) = p$. From 7, if the first is true then $H \leq M$. So suppose the second is true. We have that $HM$ is a subgroup and again from 7, $\vert H \vert \vert G : HM \vert = \vert H \cap M \vert \vert G : M \vert$. Also, $HM$ is a subgroup and $M \leq HM$. Since

    \begin{align*}
      \vert HM \vert &= \frac{\vert H \vert \vert K \vert}{\vert H \cap K \vert} \\
      &= \vert H : H \cap K \vert \vert K \vert
    \end{align*}

    we have $\vert HM = k \vert K \vert$ for some positive integer $k$, so either $\vert G : HM \vert = p$ or $\vert G : HM \vert = 1$. If it is $p$ then from 7, $\vert H = \vert H \cap M \vert$, so $H \leq M$. If instead it is $1$, then $\vert HM \vert = \vert G \vert$, so $HM = G$ and $\vert HM \vert = p \vert M \vert$. So
    $
    \vert HM \vert = \vert H : H \cap M \vert \vert K \vert
    $
    and $\vert H : H \cap M \vert = p$.
  \end{proof}

  \item \begin{proof}
    Define $\phi: G \to G/M \times G/N, a \mapsto (aN, aM)$. First we show that $\phi$ is well-defined. Suppose $g_1 = g_2 \in G/(M \cap N)$, so $g_1 = g_2m$ for some $m \in M \cap N$. Then $(g_1M, g_1N) = (g_2M, g_2N)$ so $\phi(g_1) = \phi(g_2)$.

    Next we show that $\phi$ is a homomorphism. Let $a, b \in G/(M \cap N)$. Then,
    $$
    \phi(ab) = (abN, abM) = (aM, aN)(bM, bN) = \phi(a)\phi(b)
    $$.

    For surjectivity, let $(aN, bM) \in (G/N) \times (G/M)$. Since $G = MN$, $a = m_1n_1$ and $b = m_2n_2$ for some $m_1, m_2 \in M$ and $n_1, n_2 \in N$. So $aM = m_1n_1M = m_1Mn_1 = n_1M$, so $gM = Mg$ where $g = mn$. Similarily, $bN = m_2N$. Thus $\phi(m_2n_1) = (n_1M, m_2N) = (aM, bN)$.

    Finally, note that $\phi(a) = (1M, 1N)$, with $a \in M \cap N$. So, $\operatorname{Ker}(\phi) = M \cap N$. Thus, by the First Isomorphism Theorem, since $\phi: G \to G/M \times G/N$ is a surjective homomorphism, $G/(M \cap N) \cong G/M \times G/N$.
  \end{proof}
\end{enumerate}
\end{document}
