\documentclass[11pt, letterpaper]{article}
\usepackage{fullpage}
\usepackage{amsmath,amsthm,amsfonts,amssymb,amscd}
\usepackage{lastpage}
\usepackage{enumerate}
\usepackage{fancyhdr}
\usepackage{mathrsfs}
\usepackage{enumitem} % \setlist
% for \imageans: float for [H] so the figure floats
\usepackage{graphicx}
\usepackage{adjustbox}
\usepackage{float} 

\setlength{\parindent}{0.25in}
\setlength{\parskip}{0.05in}
% indent paragraphs in list
\setlist{  
  listparindent=\parindent,
  parsep=0pt,
}

% Include graphics in answer
\newcommand{\imageans}[1]
{%
    \begin{figure}[H]
        \centering
        \includegraphics[width=0.4\linewidth]{#1}
    \end{figure}
}

% comments inside align environnment
\newcommand{\comment}[1]{%
  \text{\phantom{(#1)}} \tag{#1}
}
\newtheorem{theorem}{Theorem}
\newtheorem{lemma}{Lemma}
% Cases for Proof environment
\newlist{pcases}{enumerate}{1}
\setlist[pcases]{
  label=\underline{Case~\arabic*}:\protect\thiscase.~,
  ref=\arabic*,
  align=left,
  labelsep=0pt,
  leftmargin=0pt,
  labelwidth=0pt,
  parsep=0pt
}
\newcommand{\case}[1][]{%
  \if\relax\detokenize{#1}\relax
    \def\thiscase{}%
  \else
    \def\thiscase{~#1}%
  \fi
  \item
}

% Edit these as appropriate
\newcommand\course{Math 4120}
\newcommand\hwtitle{HW 1}                  
\newcommand\name{David Tran}
\newcommand\studentid{251169871}

\fancypagestyle{firststyle}
{
    \headheight 35pt
    \lhead{\name}
    \lhead{\name\\\studentid}
    \chead{\textbf{\LARGE \hwtitle}}
    \rhead{\course \\ \today}
    \lfoot{}
    \cfoot{}
    \rfoot{\small\thepage}
    \headsep 1.5em
}

\DeclareUnicodeCharacter{2212}{-}
\begin{document}

\thispagestyle{firststyle}

\setlist[enumerate]{leftmargin=*} % remove enuemrate indentation

\subsection*{7.3}

\begin{enumerate}
  \setcounter{enumi}{21}
  \item \begin{enumerate}
    \item \begin{proof}
      Let $a \in R$ and $A = \lbrace x \in R \mid ax = 0 \rbrace$. Let $x, y \in A$.Then $a(x - y) = ax - ay = 0$, so $A$ is a subring of $R$. Now let $r \in R$. Then $axr = 0r = 0$, so the subring is closed under right-multiplication by $R$. Thus $A$ is a right ideal of $R$.

      Let $B = \lbrace x \in R \mid xa = 0 \rbrace$. Let $x, y \in B$. Then $(x - y)a = xa - ya = 0$, so $B$ is a subring of $R$. Now let $r \in R$. Then $rxa = r0 = 0$, so the subring is closed under left-multiplication by $R$. Thus $B$ is a left ideal of $R$.
    \end{proof}

    \item \begin{proof}
      Let $L \subseteq R$ a left-ideal of $R$ and define $A = \lbrace x \in R \mid xa = 0, \forall a \in L \rbrace$. Let $x, y \in A$ and $a \in L$. Then $(x - y)a = xa - ya = 0$. So $A$ is a subring of $R$. Now let $r \in R$. Since $L$ is a left-ideal, $ra \in L$, so $xra = x(ra) = 0$. Thus the subring is closed under right-multiplication by $R$. Therefore it is a right ideal of $R$. Also, $rxa = r(xa) = 0$, so it is also closed under left-multiplication. Thus $A$ is also a left ideal, and thus a two-sided ideal of $R$.
    \end{proof}
  \end{enumerate}
  \setcounter{enumi}{28}
  \item \begin{proof}
    Let $x, y \in R$ such that $x^m = y^n = 0$. Then $(x + y)^{m+n} = \sum_{k=0}^{m+n} \binom{m+n}{k} x^k y^{m+n-k}$. Since $x^m = y^n = 0$, the sum is zero for $k \geq m$ or $k \leq n$. Thus $(x + y)^{m+n} = 0$, so $\mathfrak N(R)$ is a subring of $R$. Now let $r \in R$. Then $(rx)^m = r^m x^m = 0$, and similarily $(xr)^m = 0$, so $\mathfrak N(R)$ is closed under multiplication by $R$. Therefore $\mathfrak N(R)$ is an ideal of $R$.
  \end{proof}
  
  \item \begin{proof}
    Let $r + \mathfrak N(R) \in R/\mathfrak N(R)$ be nilpotent. Then $(r + \mathfrak N(R))^n = r^n + \mathfrak N(R) = 0 + \mathfrak N(R)$. for some positive integer $n$. Then $r \in \mathfrak N(R)$, so $r = \bar 0$. Thus only $\bar 0 + \mathfrak N(R)$ is nilpotent in $R/\mathfrak N(R)$, so $\mathfrak N(R/\mathfrak N(R))$ is trivial.
  \end{proof}
\end{enumerate}

\subsection*{7.4}

\begin{enumerate}
  \setcounter{enumi}{12}
  \item \begin{enumerate}
    \begin{proof}
      Suppose $\varphi^{-1}(P) \neq R$ and let $ab \in \varphi^{-1}(P)$. Then $\varphi(ab) = \varphi(a)\varphi(b) \in P$, so $\varphi(a) \in P$ or $\varphi(b) \in P$. Thus $a \in \varphi^{-1}(P)$ or $b \in \varphi^{-1}(P)$, so $\varphi^{-1}(P)$ is a prime ideal of $R$.

      Thus, if $R \subseteq S$ and $\varphi$ is the inclusion homomorphism, if $P$ is a prime ideal of $S$, then either $\varphi^{-1}(P) = R$, in which case $R \subseteq P$ so $R \cap P = R$, or $\varphi^{-1}(P)$ is a prime ideal of $R$, in which case $R \cap P = \varphi^{-1}(P)$; a prime ideal of $R$.
    \end{proof}
  \end{enumerate}

  \setcounter{enumi}{23}
  \item \begin{enumerate}
    \begin{proof}
      Let $R$ be a Boolean ring and $(A) \subseteq R$ an ideal of $R$ finitely generated by $A \subseteq R$. Suppose $x, y \in A$ are distinct. Consider $z = x + y + xy \in (A)$. Then $xz = x + xy + xy = x$, and similarily $yz = y$. Thus $z$ alone generates $(A)$. By induction, any number of distinct elements in $A$ can be generated by a single element. Thus $(A)$ is a principal ideal.
    \end{proof}
  \end{enumerate}

  \setcounter{enumi}{34}
  \item \begin{enumerate}
    Let $\mathcal I$ be the family of ideals in $R$ not containing $A$ and $I_1 \subseteq I_2 \subseteq \dots \subseteq I_n$ a chain of ideals in $\mathcal I$. Then $J = \bigcup_{i = 1}^n I_i$ is closed under subtraction since for any $a, b \in J$, $a, b$ is contained in some ideals $I_a, I_b$, respectively. The maximal (by inclusion) of $I_a, I_b$ thus contains both and is itself is an ideal and thus closed under subtraction. $J$ is also closed under multiplication in $R$ since each $I_i$ is closed under multiplication in $R$. So $J$ is an ideal. $J \in \mathcal I$ since if it weren't, then $J$ would contain $A$. If $J$ contains $A$, then each element of $A$ is contained in some ideal $I_i$, and the maximal (by inclusion) of all such ideals would contain all of $A$, contradicting that each $I_i \in \mathcal I$.

    So $J$ each chain of ideals in $\mathcal I$ has an upper bound in $\mathcal I$. By Zorn's lemma, $\mathcal I$ has a maximal element $M$.
  \end{enumerate}

  
\end{enumerate}

\end{document}
